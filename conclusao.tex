\chapter{CONCLUSÕES}

Este trabalho possibilitou a criação de uma documentação sucinta da área de Visão Computacional, reunindo as principais técnicas utilizadas para o desenvolvimento de aplicações na área. As pesquisas relacionadas ao desenvolvimento desses sistemas tiveram como objetivo agregar as principais ferramentas de linguagem Python utilizadas atualmente. Como resultado dessas pesquisas, descobriu-se que o OpenCV é uma ótima opção para o desenvolvimento de aplicações na área de Visão Computacional em linguagem Python, contendo diversos materiais e livros descrevendo suas funcionalidades. Do mesmo modo, as pesquisas voltadas para o \textit{framework} SimpleCV mostraram a ferramenta como um complemento que simplifica o desenvolvimento de sistemas de Visão Computacional.

Com a utilização de exemplos práticos utilizando essas ferramentas, foi possível apresentar conceitos da área de Visão Computacional normalmente vistos de forma abstrata nos principais materiais desse domínio de estudo. Com isso, o trabalho também permitiu a aproximação entre o domínio da Visão Computacional e profissionais da computação, compondo uma base de consulta para o estudo e desenvolvimento de trabalhos futuros.

Através das pesquisas para o desenvolvimento da aplicação mostrada como estudo de caso, percebeu-se que as ferramentas da área de Visão Computacional para Python estão maduras o bastante para que pessoas com interesse, na área da computação, possam utilizá-las para o desenvolvimento de aplicações.

O estudo de caso obteve sucesso na criação de uma aplicação para detecção e rastreamento de objetos circulares monocromáticos, possibilitando a recuperação de informações pertinentes do mesmo. Dessa forma, possíveis funcionalidades podem ser criadas, como por exemplo estimar o quão distante o objeto está em relação ao vídeo, baseando-se no raio do objeto circular, obtido através dos cálculos feitos pela aplicação.

A base do que foi desenvolvido para a criação do estudo de caso pode ser aplicada em projetos de maior complexidade que visam resolver problemas cotidianos. Exemplos de projetos que podem utilizar a mesma base do que foi desenvolvido são aplicados em áreas de segurança e esportes. Através da detecção de bolas em esportes como futebol e tênis por exemplo, é possível determinar quanto tempo a bola esteve em jogo, ou então concluir se a bola tocou o lado de dentro ou fora da quadra. Em áreas da segurança, mais especificamente a área de autenticação, a utilização da lógica de detecção de objetos circulares pode ser aplicada para reconhecimento do globo ocular, visando avaliá-lo biometricamente para credenciar a entrada de pessoas em áreas restritas.

Este projeto permite também que outros trabalhos futuros possam ser elaborados, como por exemplo, utilizar o Arduino\footnote{Plataforma de hardware e software livre que simplifica a criação e prototipagem de projetos de eletrônica.} visando integrar a aplicação com robôs para que possam se guiar a partir do objeto detectado pela aplicação. Ainda, estimando a distância do objeto em relação ao vídeo, pode-se criar uma inteligência capaz de permitir que o robô ande mais rápido se o objeto estiver distante do vídeo, e de mesma forma, ande mais devagar se o objeto estiver próximo.

Conclui-se, então, que o desenvolvimento de aplicações da Visão Computacional pode ser otimizado com a utilização de tecnologias atuais, possibilitando a criação de aplicações na área de Visão Computacional com maior rapidez. Dessa forma, projetos são mais facilmente desenvolvidos, popularizando ainda mais a área de Visão Computacional.
