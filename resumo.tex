\begin{center}
	\textbf{RESUMO}
\end{center}

\singlespacing

\noindent O sentido da visão nos seres humanos tem grande importância para a realização de tarefas do cotidiano. Devido a isso, desenvolver essa habilidade em sistemas de informação tem se tornado igualmente importante. Este campo de estudo denomina-se Visão Computacional. Nos últimos anos, o aumento de novas tecnologias tem gerado crescente avanço nesse campo de pesquisa. Apesar disso, a área de Visão Computacional ainda se encontra imatura, não tendo uma formulação padrão para resolução de problemas. Este trabalho propõe apresentar essa ciência, através de conceitos necessários para o seu entendimento. Também são mostradas as ferramentas que podem ser utilizadas para o desenvolvimento de aplicações utilizando a linguagem Python, apresentando técnicas comumente utilizadas no processo de implementação de sistemas na área de Visão Computacional. Ainda, através de um estudo de caso, é exposto o desenvolvimento de uma aplicação capaz de detectar e rastrear objetos circulares em vídeo. A proposta é criar um material de referência que possa ser utilizado para introdução aos estudos na área de Visão Computacional, através das ferramentas e práticas apresentadas neste trabalho. \\

\noindent PALAVRAS-CHAVE: Visão Computacional, Segmentação de imagem, Python, OpenCV
