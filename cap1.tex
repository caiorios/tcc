\chapter{INTRODUÇÃO}

Os seres humanos possuem sentidos fundamentais para a sua sobrevivência. Graças a isso, eles conseguem interagir e se comunicar com o mundo a sua volta. Nesse contexto, existe o sentido da visão, responsável por detectar imagens e transmití-las ao cérebro para que possam ser interpretadas. Essas imagens declaram a presença de objetos, cada um deles contendo cores e formas. Estes elementos possuem particularidades que os tornam únicos no universo em que estão inseridos. Por meio desses fatores, somados a outras habilidades, os seres humanos conseguem identificar e classificar esses elementos, criando um significado e atribuindo um grau de importância para cada um deles.

Pode-se exemplificar esse processo ao descrever como são identificados e classificados objetos vistos no dia-a-dia. Ao observar um carro, um dos aspectos mais facilmente idenficados é a sua cor. Outra tarefa simples é dizer o quão conservado o carro está, observando se o mesmo possui algum arranhão ou amassado. Pode-se ainda classificar esse carro ao ver se o mesmo é conversível, ou se possui um porta-malas maior que os demais. No fim desta análise, tem-se uma visão completa sobre o carro.

Por ser um dos sentidos mais importantes para os seres humanos, o interesse dos estudiosos em recriar o sistema visual em computadores se tornou uma tarefa importante. Alguns dos fatores responsáveis por tornar esse campo de estudo viável foram a diminuição do preço dos equipamentos e a evolução das tecnologias, possibilitando acesso para o desenvolvimento de pesquisas nesta área.

Apesar dessa ascensão e disseminação, estudos têm mostrado a área de Visão Computacional concernentes ao sistema de visão. \citeonline{SZELISKI} descreve que, apesar de haver técnicas confiáveis capazes de realizar tarefas desse sistema, ter um computador capaz de interpretar uma imagem no mesmo nível de percepção de uma criança é um objetivo que ainda apresenta vários desafios.

Como consequência das pesquisas e avanços tecnológicos, o amadurecimento em algoritmos de Visão Computacional tornou possível a criação de ferramentas capazes de auxiliar a implementação eficiente de sistemas nesta área. Com o uso dessas ferramentas, inúmeras contribuições científicas tem sido apresentadas para aprimorar o desenvolvimento desses sistemas.

\section{Justificativa do trabalho}

O tema de Visão Computacional foi escolhido devido ao aumento da disponibilidade de recursos que possibilitem sua aplicação para resolução de problemas atuais. Atualmente, aplicações tem sido desenvolvidas para apresentar soluções onde até então a intervenção humana era indispensável. Dessa forma, disponibilizar um material que reúne conceitos e práticas das principais técnicas da área de Visão Computacional torna o trabalho uma fonte acessível para os iniciantes na área que buscam conhecimento e aprimoramento nesse domínio de estudo.

\section{Objetivo}

Este trabalho tem como objetivo inicial apresentar os principais conceitos de Visão Computacional. A partir do entendimento desses conceitos, almeja-se explorar a utilização de algumas das principais ferramentas para o desenvolvimento de aplicações neste campo de estudo. Finalmente, através de um estudo de caso, este trabalho visa mostrar o desenvolvimento de uma aplicação capaz de detectar e rastrear objetos circulares monocromáticos\footnote{É a radiação produzida por apenas uma cor.}, com base nos conceitos e ferramentas pesquisadas.

\section{Estrutura do trabalho}

Este trabalho é composto por seis Capítulos, organizados conforme descrito abaixo.

O Capítulo 2 apresenta a área de Visão Computacional, através de conceitos e histórico sobre o assunto. São apresentados também domínios onde a Visão Computacional vem se tornando vital. Ainda é exposta superficialmente sua estrutura, descrevendo suas disciplinas.

O Capítulo 3 descreve os conceitos e técnicas relevantes para o entendimento deste trabalho. Nesse tópico, são descritas as áreas de Processamento de Imagem, Segmentação e Reconhecimento.

O Capítulo 4 descreve as principais ferramentas da área de Visão Computacional, com base nas tecnologias escolhidas. Essas ferramentas são apresentadas através de guias para instalação e uso.

O Capítulo 5 apresenta a implementação do estudo de caso, contendo detalhes das tecnologias utilizadas e os principais trechos do código-fonte. Também são apresentados os resultados obtidos a partir da aplicação desenvolvida.

Por fim, no Capítulo 6, são apresentadas as conclusões deste trabalho.
