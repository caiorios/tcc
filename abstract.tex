\begin{center}
\textbf{ABSTRACT}
\end{center}

\singlespacing

\noindent The sense of vision in humans have great importance to perform daily tasks. Because of this, the development of this skill in information systems has become equally important. This field of study is called Computer Vision. In recent years, the advance of new technologies has created an increasing breakthrough in this research field. Nevertheless, this area is still immature, lacking a standard formulation for troubleshooting. This paper aim to present this science through it's basic concepts. Tools that can be used to develop applications using the Python language are also shown, presenting techniques commonly used to implementing systems in the field of Computer Vision Systems. Yet, through a case study, the development of an application capable of detecting and tracking circular objects in video is exposed. The proposal is to create a reference material that can be used for introducing studies in this area, through the tools and practices presented in this work. \\

\noindent KEYWORDS: Computer Vision, Image Segmentation, Python, OpenCV
